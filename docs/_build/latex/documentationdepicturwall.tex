%% Generated by Sphinx.
\def\sphinxdocclass{report}
\documentclass[letterpaper,10pt,french]{sphinxmanual}
\ifdefined\pdfpxdimen
   \let\sphinxpxdimen\pdfpxdimen\else\newdimen\sphinxpxdimen
\fi \sphinxpxdimen=.75bp\relax
\ifdefined\pdfimageresolution
    \pdfimageresolution= \numexpr \dimexpr1in\relax/\sphinxpxdimen\relax
\fi
%% let collapsable pdf bookmarks panel have high depth per default
\PassOptionsToPackage{bookmarksdepth=5}{hyperref}

\PassOptionsToPackage{warn}{textcomp}
\usepackage[utf8]{inputenc}
\ifdefined\DeclareUnicodeCharacter
% support both utf8 and utf8x syntaxes
  \ifdefined\DeclareUnicodeCharacterAsOptional
    \def\sphinxDUC#1{\DeclareUnicodeCharacter{"#1}}
  \else
    \let\sphinxDUC\DeclareUnicodeCharacter
  \fi
  \sphinxDUC{00A0}{\nobreakspace}
  \sphinxDUC{2500}{\sphinxunichar{2500}}
  \sphinxDUC{2502}{\sphinxunichar{2502}}
  \sphinxDUC{2514}{\sphinxunichar{2514}}
  \sphinxDUC{251C}{\sphinxunichar{251C}}
  \sphinxDUC{2572}{\textbackslash}
\fi
\usepackage{cmap}
\usepackage[T1]{fontenc}
\usepackage{amsmath,amssymb,amstext}
\usepackage{babel}



\usepackage{tgtermes}
\usepackage{tgheros}
\renewcommand{\ttdefault}{txtt}



\usepackage[Sonny]{fncychap}
\ChNameVar{\Large\normalfont\sffamily}
\ChTitleVar{\Large\normalfont\sffamily}
\usepackage{sphinx}

\fvset{fontsize=auto}
\usepackage{geometry}


% Include hyperref last.
\usepackage{hyperref}
% Fix anchor placement for figures with captions.
\usepackage{hypcap}% it must be loaded after hyperref.
% Set up styles of URL: it should be placed after hyperref.
\urlstyle{same}

\addto\captionsfrench{\renewcommand{\contentsname}{Panel invité:}}

\usepackage{sphinxmessages}
\setcounter{tocdepth}{2}



\title{Documentation de PicturWall}
\date{juin 10, 2021}
\release{1.16}
\author{Gatien DUBOC}
\newcommand{\sphinxlogo}{\vbox{}}
\renewcommand{\releasename}{Version}
\makeindex
\begin{document}

\ifdefined\shorthandoff
  \ifnum\catcode`\=\string=\active\shorthandoff{=}\fi
  \ifnum\catcode`\"=\active\shorthandoff{"}\fi
\fi

\pagestyle{empty}
\sphinxmaketitle
\pagestyle{plain}
\sphinxtableofcontents
\pagestyle{normal}
\phantomsection\label{\detokenize{index::doc}}


\sphinxAtStartPar
Bienvenue sur la Documentation de PicturWall !

\sphinxAtStartPar
Ici, vous retrouvez la plupart des fonctionnalités de PicturWall, mais aussi comment les utiliser. Vous allez y trouver des fonctionnalitées basique comme la mise en ligne de médias mais également des fonctions bien plus avancées comme le partage d’imprimante via PicturWall.
\begin{description}
\item[{Cette documentation sera regroupée en 3 grands thèmes:}] \leavevmode\begin{itemize}
\item {} 
\sphinxAtStartPar
{\hyperref[\detokenize{panel_invite/index:panel-invite}]{\sphinxcrossref{\DUrole{std,std-ref}{Le panel invité}}}}: Présentation et utilisation du panel destiné aux convives.

\item {} 
\sphinxAtStartPar
\sphinxstylestrong{Le panel animateur}: Votre outil ! Celui qui va vous permettre de gérer PicturWall.

\item {} 
\sphinxAtStartPar
\sphinxstylestrong{Le diaporama}: Présentation des différents modules du diaporama.

\end{itemize}

\end{description}


\chapter{Présentation du panel invité}
\label{\detokenize{panel_invite/index:presentation-du-panel-invite}}\label{\detokenize{panel_invite/index:panel-invite}}\label{\detokenize{panel_invite/index::doc}}
\sphinxAtStartPar
Le panel invité, c’est la page que les convives de la soirée vont utiliser pour mettre en ligne leurs médias.
Elle est très facile d’utilisation.

\sphinxAtStartPar
Dans la suite de cette page, je vais utiliser le pronom « vous ». Cela désigne toute personne voulant envoyer un média sur PicturWall. Ce n’est pas uniquement réservé à l’animateur.


\chapter{Comment se rendre sur le panel invité ?}
\label{\detokenize{panel_invite/index:comment-se-rendre-sur-le-panel-invite}}
\sphinxAtStartPar
Afin de ce connecter au panel invité, vous devrez être préalablement connecté au réseau wifi public de PicturWall.
Ensuite, vous allez ouvrir votre navigateur préféré, puis tapez l’adresse URL suivante: \sphinxhref{http://picturwall.tv/}{picturwall.tv}

\sphinxAtStartPar
Bien\sphinxhyphen{}sûr, c’est exactement la même chose sur téléphone.
Vous ouvrez votre navigateur et tapez \sphinxhref{http://picturwall.tv/}{picturwall.tv} dans votre barre de recherche.


\section{Inscription}
\label{\detokenize{panel_invite/index:inscription}}
\sphinxAtStartPar
Pour s’incrire sur le panel invité, vous devez simplement remplir votre nom et prénom.
Ensuite, vous acceptez les conditions d’utilisations, et vous cliquez sur \sphinxincludegraphics{{invite_bouton_envoyer}.PNG}.

\begin{figure}[htbp]
\centering

\noindent\sphinxincludegraphics{{invite_inscription}.PNG}
\end{figure}


\chapter{Envoyer un média}
\label{\detokenize{panel_invite/index:envoyer-un-media}}\label{\detokenize{panel_invite/index:invite-media}}
\sphinxAtStartPar
L’envoi de média sur PicturWall est très simple:
\begin{itemize}
\item {} 
\sphinxAtStartPar
Sélectionner son média dans votre galerie photo, ou de prendre la photo en direct !

\item {} 
\sphinxAtStartPar
Écrire un {\hyperref[\detokenize{panel_invite/index:invite-media-commentaire}]{\sphinxcrossref{\DUrole{std,std-ref}{Commentaire}}}} (facultatif) ;)

\item {} 
\sphinxAtStartPar
Cocher (ou non) la case pour {\hyperref[\detokenize{panel_invite/index:invite-media-imprimer}]{\sphinxcrossref{\DUrole{std,std-ref}{« imprimer le souvenir »}}}} !

\end{itemize}

\begin{figure}[htbp]
\centering

\noindent\sphinxincludegraphics{{invite_medias}.PNG}
\end{figure}


\section{Commentaire}
\label{\detokenize{panel_invite/index:commentaire}}\label{\detokenize{panel_invite/index:invite-media-commentaire}}
\begin{figure}[htbp]
\centering

\noindent\sphinxincludegraphics{{invite_medias_commentaire}.PNG}
\end{figure}

\sphinxAtStartPar
Avec PicturWall, vous avez la possibilité d’ajouter un commentaire avec l’envoi de votre souvenir.
Celui\sphinxhyphen{}ci s’affichera sur le diaporama (ajouter lien), en haut de l’écran.

\sphinxAtStartPar
Le commentaire doit être composé d’au moins 5 caractères.

\sphinxAtStartPar
De plus, vous pouvez y ajouter un smiley, via l’icone associé à droite de la barre de commentaire \sphinxincludegraphics{{invite_bouton_smiley}.PNG}.

\sphinxAtStartPar
Pour finir, vous pouvez personaliser la couleur du commentaire envoyé grâce au sélecteur de couleur \sphinxincludegraphics[scale=0.75]{{invite_bouton_commentaire_couleur}.PNG}.

\begin{sphinxadmonition}{note}{Personnalisation}

\sphinxAtStartPar
Vous pouvez changer la couleur par défaut des commentaires via un paramètre du panel animateur (ajouter lien).
\end{sphinxadmonition}


\section{Imprimer le souvenir}
\label{\detokenize{panel_invite/index:imprimer-le-souvenir}}\label{\detokenize{panel_invite/index:invite-media-imprimer}}
\begin{figure}[htbp]
\centering

\noindent\sphinxincludegraphics{{invite_medias_imprimer}.PNG}
\end{figure}

\begin{sphinxadmonition}{important}{Important:}
\sphinxAtStartPar
Pour que cette case soit visible, il faut activer le service d’impression (ajouter lien).
\end{sphinxadmonition}

\sphinxAtStartPar
Grâce à PicturWall, les invités peuvent eux\sphinxhyphen{}mêmes imprimer leur souvenir !

\sphinxAtStartPar
Bien sûr, tout est prévu sur le panel animateur (ajouter lien) pour que vous puissiez définir des limites d’impressions par utilisateurs, impresssions maximales… (ajouter lien).

\begin{sphinxadmonition}{note}{Autoriser les invités à imprimer}

\sphinxAtStartPar
Pour que cette case soit active, il faut autoriser les invités à imprimer (ajouter lien).
\end{sphinxadmonition}


\chapter{Imprimer après coup}
\label{\detokenize{panel_invite/index:imprimer-apres-coup}}\label{\detokenize{panel_invite/index:invite-impression-apres-coup}}
\sphinxAtStartPar
Vous pouvez aussi vouloir imprimer un média après coup !

\sphinxAtStartPar
Pour cela, vous devez cliquer sur ce bouton: \sphinxincludegraphics[scale=0.75]{{invite_bouton_impression}.PNG}

\sphinxAtStartPar
Vous allez donc atterir sur cette page:

\begin{figure}[htbp]
\centering

\noindent\sphinxincludegraphics{{invite_impressions}.PNG}
\end{figure}

\sphinxAtStartPar
Ensuite, il ne vous restera plus qu’à cliquer sur le média que vous voulez imprimer:

\begin{figure}[htbp]
\centering

\noindent\sphinxincludegraphics{{invite_impressions_imprimer_medias}.PNG}
\end{figure}

\sphinxAtStartPar
Et vous pouvez vous rendre jusqu’à l’imprimante pour récupérer votre média ;)

\begin{sphinxadmonition}{note}{Autoriser les invités à imprimer après coup}

\sphinxAtStartPar
Pour que les invités puissent accéder à cette page, il faut autoriser l’impression après coup (ajouter lien).
\end{sphinxadmonition}

\sphinxAtStartPar
Afin de revenir à la page d’envoi de médias, il faut cliquer sur ce bouton: \sphinxincludegraphics[scale=0.75]{{invite_impressions_bouton_imprimer_souvenir}.PNG}


\section{Média en or}
\label{\detokenize{panel_invite/index:media-en-or}}
\sphinxAtStartPar
Sur PicturWall, il existe un concept que l’on nomme « \sphinxstylestrong{média en or} ».

\sphinxAtStartPar
C’est en réalité un ou plusieurs médias que l’animateur met à disposition de tous pour l’impression. Nous détaillons cette fonction dans le panel animateur, page « Édition de médias » (ajouter lien).

\sphinxAtStartPar
Les invités voient les médias mis à disposition de cette façon:

\begin{figure}[htbp]
\centering

\noindent\sphinxincludegraphics{{invite_impressions_media_or}.PNG}
\end{figure}

\sphinxAtStartPar
Vous pouvez imprimer ce type de média comme tous les autres.

\begin{sphinxadmonition}{note}{Médias en or}

\sphinxAtStartPar
Les médias en or sont tous sélectionnés par l’animateur. Par exemple, cela peut\sphinxhyphen{}être des photos d’autres invités ou du photographe.
\end{sphinxadmonition}



\renewcommand{\indexname}{Index}
\printindex
\end{document}